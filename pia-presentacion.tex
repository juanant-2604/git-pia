\documentclass[10pt]{article}

\usepackage{newpxmath,newpxtext}
\usepackage[english,spanish]{babel}
\usepackage{latexsym,graphicx}
\usepackage{hyperref}

%%% Código de Computadora

\usepackage{listings}

\lstset{
  language = Prolog,
  basicstyle = \scriptsize \ttfamily,
  numbers = left,
  numberstyle = \tiny
}

%%% Teclado y Lenguaje en español

\usepackage[utf8]{inputenc}
\usepackage[T1]{fontenc}
\usepackage{url}
\selectlanguage{spanish}

\begin{document}

\title{\textbf{Programación para la Inteligencia Artificial} \\
  Presentación del curso}
\author{\textbf{Dr. Alejandro Guerra-Hernández} \\
  Maestría en Inteligencia Artificial \\
  Universidad Veracruzana \\
  CIIA -- Centro de Investigación en Inteligencia Artificial \\
  Sebastián Camacho
  No 5, Xalapa, Ver., México 91000 \\ \texttt{aguerra@uv.mx} \\
  \url{http://www.uv.mx/personal/aguerra/}}

\date{13 de Agosto del 2019}

\maketitle

Bienvenidos al curso Programación para la Inteligencia Artificial. La
Inteligencia Artificial (IA) ha producido lenguajes de programación
que son a la vez, herramientas y sujeto de estudio de esta
disciplina. Este curso introduce los paradigmas de Programación
Lógica, Funcional (y Orientado a Agentes) desde esta perspectiva. Se
introducen los fundamentos teóricos de ambos paradigmas y se explora
su uso en problemas característicos de la IA, i.e., estrategias de
resolución de problemas, búsquedas heurísticas, inducción de árboles
de decisión, etc., utilizando los lenguajes de programación Prolog,
Lisp. Se utiliza el problema de inducción de árboles de decisión para
comparar los lenguajes de programación aquí introducidos.

El curso tiene una duración de 60 horas; se oferta durante un
semestre, organizado en dos sesiones presenciales de dos horas de
duración. 

\section{Objetivos}
\begin{itemize}
\item El estudiante identificará los conceptos fundamentales de la
  Programación Lógica y la Programación Funcional, así como sus
  origenes en la IA.
\item El estudiante adquirirá las habilidades para resolver problemas
  complejos con el lenguaje de Programación Lógica Prolog.
\item El estudiante adquirirá las habilidades para resolver problemas
  complejos con el lenguaje de Programación Funcional Lisp.
\item El estudiante podrá generalizar estas habilidades usando otros
  lenguajes de programación.
\end{itemize}

\section{Evaluación}

La nota final del curso será calculada de la siguiente forma:

\begin{itemize}
\item Las tareas/Ejercicios cubren un 50\% de la nota final.
\item El examen de conocimientos cubre el 20\% de la nota final.
\item El proyecto integrador cubre el 30\% de la nota final.
\end{itemize}

Para obtener una \textbf{nota aprobatoria} en el curso, el alumno
deberá haber obtenido notas aprobatorias en cada uno de los elementos
de evaluación (aprobar todas sus tareas, examen y proyecto).

\subsection{Tareas}

Las tareas pueden requerir investigación bibliográfica, ejercicios
teóricos, experimentación en la computadora y/o exposiciones frente a
grupo.  La entrega y evaluación de éstas se realizará conforme a los
siguientes lineamentos:

\begin{description}
\item[Entrega.] Las tareas se entregan al inicio de clase del día
  designado para ello. El mérito de las mismas decrece 25\% por cada
  24 horas de retraso. El calendario del curso marca las fechas
  tentativas para cada tarea, su entrega suele ser dos semanas más
  tarde.
\item[Formato.] Las tareas se procesan con \LaTeX, siguiendo la
  plantilla de este mismo documento. En todas las partes que
  involucran código de computadora o algoritmos, éste deberá ser
  documentado apropiadamente (Ver ejemplo del
  Cuadro~\ref{tab:ejPrograma}).

  \begin{table}
\begin{lstlisting}[language=Prolog]
%%% allPerms computa todas las permutaciones de los elementos
%%% de la lista L en la lista L2. Ejemplo de llamada:
%%% ?- allPerms([1,2,3],Resp).
%%% Resp = [[1, 2, 3], [1, 3, 2], [2, 1, 3], [2, 3, 1], [3, 1, 2],
%%% [3, 2, 1]].

    allPerms(L,L2) :-
      findall(Perm,permutation(L,Perm),L2).
\end{lstlisting}
    \caption{Ejemplo de código de computadora y sus comentarios (Prolog).}
    \label{tab:ejPrograma}
  \end{table}
  
\item[Evaluación.] Para que un ejercicio o pregunta de la tarea reciba
  puntaje alguno, más del 50\% deberá estar resuelta de manera
  correcta. Es más redituable invertir el tiempo en contestar una
  pregunta de manera correcta y completa, que responder a dos de
  manera incompleta.
\end{description}

Cualquier forma de \textbf{plagio} causa la expulsión definitiva del
curso. Esto incluye: Reportar trabajo de otros como propio y no citar
pertinentemente las referencias usadas.

\subsection{Examen}

Se trata de una evaluación de los conocimientos teóricos del curso. Su
fecha de aplicación aparece en el calendario de este documento.

\subsection{Proyecto integrador}

Se trata de un proyecto práctico que requiere la aplicación las
diversas técnicas introducidas en el curso, posiblemente complementado
con el contenido de otros cursos de este semestre. Se comienza a
definir a la mitad del mismo para tratar de integrar los intereses de
investigación de los estudiantes. Las fechas de revisiones parciales y
final del proyecto integrador aparecen el en calendario de este
documento.

\section{Material del curso}

Como es usual en nuestros cursos, el material de clases (notas,
presentaciones, código) está disponible en mi página personal:

\begin{center}
  \url{http://www.uv.mx/personal/aguerra/pia/}
\end{center}

La bibliografía básica está a disposición de ustedes en la biblioteca
del departamento ó en mi oficina. Algunos de los artículos están
disponibles en versión electrónica gracias a las suscripciones de
nuestra biblioteca central y del Conacyt.

Utilizaremos el Mundo de Tarski [1], un popular programa para explorar
el uso de la lógica de primer orden. Más información
sobre este programa puede encontrarse en:

\begin{center}
  \url{https://www.gradegrinder.net/Products/tw-index.html}
\end{center}

Haremos uso de los siguientes lenguajes de programación, todos ellos
tiene versiones para diversos sistemas operativos:

\begin{itemize}
\item \url{http://www.swi-prolog.org}
\item \url{http://www.sbcl.org}
\item \url{http://www.lispworks.com}
\item \url{http://jason.sourceforge.net/wp/}
\end{itemize}

\section{Calendario}

Este año las sesiones se llevarán a cabo los martes y jueves de 10:00
a 12:00 hrs., en el salón A-01. Las sesiones se organizarán como
sigue:

\begin{center}
  \begin{tabular}{cll}
    \textbf{Fecha} & \textbf{Tema} & \textbf{Tarea} \\
    \hline
    13/08/2018 & Presentación del curso & \\
    15/08/2018 & Introducción a los Paradigmas Lógico y Funcional & \\
    20/08/2018 & Programación Lógica, Lógica de Primer Orden & \\
    22/08/2018 & Programación Lógica, Programas Definitivos & \\
    27/08/2018 & Programación Orientada a Agentes I & \\
    29/08/2018 & &  \\
    03/09/2018 & Programación Lógica, Resolución & T1 \\
    05/09/2018 & Prolog, Conceptos Básicos I & \\
    10/09/2018 & &  \\
    12/09/2018 & Prolog, Búsquedas de Soluciones I & \\
    17/09/2018 & & \\
    19/09/2018 & Prolog, Inducción de Arboles de Decisión I & \\
    24/09/2018 & & T2 \\
    26/09/2018 & Programación Funcional,  Cálculo-$\lambda$ & \\
    01/10/2018 &  & \\
    03/10/2018 & Lisp, Conceptos Básicos I & \\
    08/10/2018 & & \\
    10/10/2018 & Lisp, Listas & \\
    15/10/2018 & Lisp, Macros & T3 \\
    17/10/2018 & & \\
    22/10/2018 & Lisp, GPS & \\
    24/10/2018 & & \\
    29/10/2018 & Lisp, Bioinformática & \\
    31/10/2018 & & \\ 
    05/11/2018 & Lisp, Inducción de Arboles de Decisión & \\
    07/11/2018 & & \\
    12/11/2018 & &  T4 \\
    14/11/2018 & Definición Proyectos Integradores & \\
    19/11/2018 & Otros Lenguajes I & \\
    21/11/2019 & & \\
    26/11/2019 & Otros Lenguajes II & \\
    28/11/2019 & & \\
    09/01/2020 & Evaluación proyecto final & \\
    \hline
  \end{tabular}
\end{center}

\section*{Referencias}

\begin{enumerate}
\bibitem{Barker-Plummer2008} {\sc Barker-Plummer, D., Barwise, J.,
    Etchemendy, J., and Liu, A.}  \newblock {\em Tarski's World,
    Revised and Expanded}, \newblock CSLI Publications, Stanford, CA.,
  USA, 2008.

\bibitem{Bordini2007a} {\sc Bordini, R.~H., H{\"u}bner, J.~F., and
    Wooldridge, M.}  \newblock {\em Programming Multi-Agent Systems in
    Agent-Speak using Jason}.  \newblock John Wiley \& Sons Ltd, 2007.

\bibitem{Bratko2012} {\sc I.~Bratko.}  \newblock {\em Prolog
    programming for Artificial Intelligence}.  \newblock Pearson,
  fourth edition, 2012.

\bibitem{Clocksin2003} {\sc Clocksin, W.~F., and Melish, C.~S.}
  \newblock {\em Programming in Prolog, using the ISO standard}.
  \newblock Springer-Verlag, Berlin-Heidelberg, Germany, 2003.

\bibitem{Graham1996} {\sc P.~Graham.}  \newblock {\em ANSI Common
    Lisp}.  \newblock Prentice Hall Series in Artificial
  Intelligence. Prentice Hall International, 1996.

\bibitem{Kluge2005} {\sc W.~Kluge.}  \newblock {\em Abstract Computing
    Machines: A Lambda Calculus Perspective}.  \newblock
  Springer-Verlag, Berlin Heidelberg New York, 2005.

\bibitem{Nilsson2000} {\sc U.~Nilsson and J.~Maluszynski.}  \newblock
  {\em Logic, Programming and Prolog}.  \newblock John Wiley \& Sons
  Ltd, 2nd edition, 2000.

\bibitem{Norvig1992} {\sc P.~Norvig.}  \newblock {\em Paradigms of
    Artificial Intelligence Programming: Case Studies in Common Lisp}.
  \newblock Morgan Kauffman Publishers, 1992.

\bibitem{Seibel2005} {\sc P.~Seibel.}  \newblock {\em Practical Common
    Lisp}.  \newblock Apress, USA, 2005.
\end{enumerate}

\vspace{\baselineskip}
\begin{flushright}\noindent
  Xalapa, Ver., México \hfill {\it Alejandro}\\
  Agosto 2019 \hfill {\it Guerra-Hernández}\\
\end{flushright}
\end{document}
